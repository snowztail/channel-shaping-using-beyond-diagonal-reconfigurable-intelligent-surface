\documentclass[journal]{IEEEtran}
% \documentclass[journal,12pt,onecolumn,draftclsnofoot]{IEEEtran}

\usepackage[table]{xcolor}
\usepackage{adjustbox}
\usepackage{algorithm}
\usepackage{algpseudocode}
\usepackage{amsfonts}
\usepackage{amsmath}
\usepackage{amssymb}
\usepackage{amsthm}
\usepackage{bookmark}
\usepackage{booktabs}
\usepackage[makeroom]{cancel}
\usepackage[american]{circuitikz}
\usepackage{cite}
\usepackage{fixmath}
\usepackage[acronym]{glossaries-extra}
\usepackage{hyperref}
\usepackage{import}
\usepackage{mathtools}
\usepackage{microtype}
\usepackage[short]{optidef}
\usepackage{pgfplots}
\usepackage{ragged2e}
\usepackage[subtle]{savetrees}
\usepackage{siunitx}
\usepackage{stfloats}
\usepackage[caption=false,font=footnotesize,subrefformat=parens,labelformat=parens]{subfig}
\usepackage{tabularx}
\usepackage{tikz}

% page limit hacks
% \usepackage{setspace}
% ! \usepackage[top=1cm, bottom=1cm, left=1cm, right=1cm]{geometry}
% \abovedisplayskip=1mm
% \belowdisplayskip=1mm
% \abovedisplayshortskip=1mm
% \belowdisplayshortskip=1mm
% \setlength{\jot}{0.1mm}
% \setlength{\floatsep}{1mm}
% \setlength{\textfloatsep}{1mm}
% \setlength{\intextsep}{1mm}
% \setlength{\skip\footins}{2mm}


% amsthm
\newtheorem{proposition}{Proposition}
\newtheorem{remark}{Remark}

% PGF/TikZ
\usetikzlibrary{arrows,calc,matrix,patterns,plotmarks,positioning,shapes}
\usetikzlibrary{decorations.pathmorphing,decorations.pathreplacing,decorations.shapes,shapes.geometric}
\usepgfplotslibrary{groupplots,patchplots}
\pgfplotsset{compat=newest}

% tabularx, ragged2e
\newcolumntype{L}{>{\RaggedRight}X}
\newcolumntype{C}{>{\centering\arraybackslash}X}
\renewcommand\tabularxcolumn[1]{m{#1}}

% algpseudocode
\makeatletter
\renewcommand{\fnum@algorithm}{\fname@algorithm{} \thealgorithm:}
\newcommand\setalgorithmcaptionfont[1]{%
	\let\my@floatc@ruled\floatc@ruled          % save \floatc@ruled
	\def\floatc@ruled{%
		\global\let\floatc@ruled\my@floatc@ruled % restore \floatc@ruled
		#1\floatc@ruled}}
\makeatother

\algrenewcommand{\algorithmicrequire}{\textbf{Input:}}
\algrenewcommand{\algorithmicensure}{\textbf{Output:}}
\algrenewcommand{\algorithmicwhile}{\textbf{While}}
\algrenewcommand{\algorithmicend}{\textbf{End}}
\algrenewcommand{\algorithmicrepeat}{\textbf{Repeat}}
\algrenewcommand{\algorithmicuntil}{\textbf{Until}}
\algrenewcommand{\algorithmicdo}{}

% glossaries-extra
\glsdisablehyper
\setabbreviationstyle[acronym]{long-short}
% \newacronym{af}{AF}{Amplify-and-Forward}
\newacronym{bd}{BD}{Beyond-Diagonal}
\newacronym{ris}{RIS}{Reconfigurable Intelligent Surface}

\begin{document}
\title{Channel Shaping Using Reconfigurable Intelligent Surfaces: From Diagonal to Beyond}
\author{
	\IEEEauthorblockN{
		Yang~Zhao,~\IEEEmembership{Member,~IEEE,}
		Hongyu~Li,~\IEEEmembership{Graduate Student Member,~IEEE,}\\
		Yijie~Mao,~\IEEEmembership{Member,~IEEE,}
		Shanpu~Shen,~\IEEEmembership{Member,~IEEE,}
		and~Bruno~Clerckx,~\IEEEmembership{Fellow,~IEEE}
	}
	% \thanks{
	% 	The authors are with the Department of Electrical and Electronic Engineering, Imperial College London, London SW7 2AZ, U.K. (e-mail: \{yang.zhao18, b.clerckx\}@imperial.ac.uk).
	% 	B. Clerckx is also with Silicon Austria Labs (SAL), Graz A-8010, Austria.
	% }
}
\maketitle

% \begin{abstract}
% \end{abstract}

% \begin{IEEEkeywords}
% \end{IEEEkeywords}

\glsresetall

\begin{section}{Assumption}
	All proposals in this paper based on assumption of \emph{asymmetric} passive \gls{bd} \gls{ris}, i.e., symmetry constraint $\mathbf{\Theta}_g = \mathbf{\Theta}_g^\mathsf{T}$ is relaxed.
	This is feasible when asymmetric passive components (e.g., ring hybrids and branch-line hybrids) \cite{Ahn2006} are available.
	This assumption was also made in Hongyu's papers \cite{Li2023b,Li2023c}.
	For quadratic problems, the proposed algorithms may be extended to symmetric \gls{bd} \gls{ris} by replacing singular value decomposition with Takagi factorization \cite{Horn2012}.
\end{section}

\begin{section}{Point-to-Point MIMO}
	\begin{subsection}{Channel Power Maximization}
		Consider a \gls{bd} \gls{ris} with $N_\mathrm{s}$ elements, which is divided into $G$ groups of equal $L$ elements.
		\begin{maxi!}
			{\scriptstyle{\mathbf{\Theta}}}{\left\lVert \mathbf{H}^\mathrm{D} + \sum_g\nolimits \mathbf{H}_g^\mathrm{B} \mathbf{\Theta}_g \mathbf{H}_g^\mathrm{F} \right\rVert _\mathrm{F}^2}{\label{op:channel_power}}{}
			\addConstraint{\mathbf{\Theta}_g^\mathsf{H} \mathbf{\Theta}_g=\mathbf{I}_L, \quad \forall g \in \mathcal{G} \triangleq \{1,\ldots,G\}}{}{}
		\end{maxi!}
		For \emph{symmetric} BD-RIS, the problem has been solved in
		\begin{itemize}
			\item Matteo's paper \cite{Nerini2023}: SISO and equivalent\footnote{Single-stream MIMO with given precoder and combiner.};
			\item Ignacio's paper \cite{Santamaria2023}: SISO and directless MISO/SIMO.
		\end{itemize}

		\begin{remark}
			% The difficulty of \eqref{op:channel_power} is that the \gls{ris} needs to balance the (additive) direct-indirect eigenspace alignment and the (multiplicative) forward-backward eigenspace alignment.
			The difficulty of \eqref{op:channel_power} is that the \gls{ris} needs to balance the additive (direct-indirect) and multiplicative (backward-forward) eigenspace alignment.
			% align the direct and indirect eigenspaces while preserving the
			Interestingly, it has the same form as the \emph{weighted orthogonal Procrustes problem} \cite{Gower2004}:
			\begin{mini!}
				{\scriptstyle{\mathbf{\Theta}}}{\lVert \mathbf{C} - \mathbf{A \Theta B} \rVert _\text{F}^2}{\label{op:weighted_orthogonal_procrustes}}{}
				\addConstraint{\mathbf{\Theta}^\mathsf{H} \mathbf{\Theta}=\mathbf{I}}{}{}
			\end{mini!}
			There exists no trivial solution to \eqref{op:weighted_orthogonal_procrustes}.
			One lossy transformation, by moving $\mathbf{\Theta}$ to one side \cite{Bell2003}, formulates a standard orthogonal Procrustes problem:
			\begin{mini!}
				{\scriptstyle{\mathbf{\Theta}}}{\lVert \mathbf{A}^\dagger \mathbf{C} - \mathbf{\Theta B} \rVert _\text{F}^2}{\label{op:standard_orthogonal_procrustes}}{}
				\addConstraint{\mathbf{\Theta}^\mathsf{H} \mathbf{\Theta}=\mathbf{I}}{}{}
			\end{mini!}
			\eqref{op:standard_orthogonal_procrustes} has a global optimal solution $\mathbf{\Theta}^\star = \mathbf{U} \mathbf{V}^\mathsf{H}$, where $\mathbf{U}$ and $\mathbf{V}$ are left and right singular matrix of $\mathbf{\mathbf{A}^\dagger \mathbf{C} \mathbf{B}^\mathsf{H}}$ \cite{Golub2013}.
			This low-complexity solution will be compared with the one proposed later.
		\end{remark}

		Inspired by \cite{Nie2017}, we propose an iterative algorithm to solve \eqref{op:channel_power}.
		The idea is to successively approximate the quadratic objective with a sequence of affine functions and solve the resulting subproblems in closed form.

		\begin{proposition}
			Start from any $\mathbf{\Theta}^{(0)}$, the sequence
			\begin{equation}
				\mathbf{\Theta}_g^{(r+1)} = \mathbf{U}_g^{(r)} \mathbf{V}_g^{(r)}, \quad \forall g \in \mathcal{G}
			\end{equation}
			converges to a stationary point of \eqref{op:channel_power}, where $\mathbf{U}_g^{(r)}$ and $\mathbf{V}_g^{(r)}$ are left and right singular matrix of
			\begin{equation}
				\begin{split}
					\mathbf{M}_g^{(r)}
					& = {\mathbf{H}_g^\mathrm{B}}^\mathsf{H} \mathbf{H}^\mathrm{D} {\mathbf{H}_g^\mathrm{F}}^\mathsf{H} + \sum_{g' < g} {\mathbf{H}_{g'}^\mathrm{B}}^\mathsf{H} \mathbf{H}_{g'}^\mathrm{B} \mathbf{\Theta}_{g'}^{(r+1)} \mathbf{H}_{g'}^\mathrm{F} {\mathbf{H}_{g'}^\mathrm{F}}^\mathsf{H} \\
					& \quad + \sum_{g' \ge g} {\mathbf{H}_{g'}^\mathrm{B}}^\mathsf{H} \mathbf{H}_{g'}^\mathrm{B} \mathbf{\Theta}_{g'}^{(r)} \mathbf{H}_{g'}^\mathrm{F} {\mathbf{H}_{g'}^\mathrm{F}}^\mathsf{H}.
				\end{split}
			\end{equation}
		\end{proposition}

		\begin{proof}
			TODO
		\end{proof}

		% For any $\mathbf{\Theta}$, define the coefficient matrix of group $g \in \mathcal{G}$ as
		% \begin{equation}
		% 	\mathbf{M}_g \triangleq
		% \end{equation}
		% \begin{
		% \begin{itemize}
		% 	\item It becomes orthogonal procrustes problem with {\color{blue}closed-form solution if $\mathbf{\Theta}$ can be moved to one side}
		% 	\item For example, if $\mathbf{A}$ is unitary, then $\arg \min_\mathbf{\Theta} \lVert \mathbf{C} - \mathbf{A \Theta B} \rVert_\text{F}^2 = \arg \min_\mathbf{\Theta} \lVert \mathbf{A}^\mathsf{H} \mathbf{C} - \mathbf{\Theta B} \rVert_\text{F}^2$
		% \end{itemize}
		% The problem is nontrivial because
		% \begin{itemize}
		% 	\item $\mathbf{H}_{\text{B},g} \triangleq \mathbf{H}_\text{B}[:, (g-1)Q+1:gQ]$, $\mathbf{H}_{\text{F},g} \triangleq \mathbf{H}_\text{F}[(g-1)Q+1:gQ, :]$
		% 	\item RIS needs to balance $\lVert \mathbf{H}_\text{B} \mathbf{\Theta} \mathbf{H}_\text{F} \rVert^2$ and $\langle \mathbf{H}_\text{D}, \mathbf{H}_\text{B} \mathbf{\Theta} \mathbf{H}_\text{F} \rangle$
		% 	\item Same form as \emph{the weighted orthogonal Procrustes problem}, no closed-form solution
		% \end{itemize}
	\end{subsection}
\end{section}


\bibliographystyle{IEEEtran}
\bibliography{library.bib}
\end{document}
