\documentclass[journal]{IEEEtran}

\usepackage{adjustbox}
\usepackage{algorithm}
\usepackage{algpseudocode}
\usepackage{amsfonts}
\usepackage{amsmath}
\usepackage{amssymb}
\usepackage{amsthm}
\usepackage{booktabs}
\usepackage{cases}
\usepackage{cite}
\usepackage[acronym]{glossaries-extra}
\usepackage{hyperref}
\usepackage{mathtools}
\usepackage[nopatch=footnote]{microtype}
\usepackage[short]{optidef}
\usepackage{pgfplots}
\usepackage[subtle]{savetrees}
\usepackage{siunitx}
\usepackage[caption=false,font=footnotesize,subrefformat=parens,labelformat=parens]{subfig}
\usepackage{tabularx}
\usepackage{tikz}
\usepackage{epstopdf}
\usepackage{multirow}
\usepackage{enumitem}
\usepackage{xr-hyper}

% amsmath
\allowdisplaybreaks
\DeclareMathOperator{\diag}{diag}
\DeclareMathOperator{\tr}{tr}
\DeclareMathOperator{\sv}{sv}
% \DeclareMathOperator{\card}{card}
\DeclareMathOperator{\conv}{conv}
\DeclareMathOperator{\rank}{rank}
\DeclareMathOperator{\ran}{ran}
% \DeclareMathOperator{\ker}{ker}

% amsthm
\theoremstyle{plain}
\newtheorem{proposition}{Proposition}
\newtheorem{corollary}{Corollary}[proposition]
\newtheorem{lemma}{Lemma}

\theoremstyle{definition}
\newtheorem{definition}{Definition}
\newtheorem{example}{Example}
\newtheorem{remark}{Remark}

% enumitem
\setlist[itemize]{leftmargin=5mm}
\setlist[enumerate]{leftmargin=5mm}

% PGF/TikZ
\usetikzlibrary{arrows,calc,matrix,patterns,plotmarks,positioning,shapes}
\usetikzlibrary{decorations.pathmorphing,decorations.pathreplacing,decorations.shapes,shapes.geometric}
\usepgfplotslibrary{groupplots,patchplots}
\pgfplotsset{compat=1.18}
\pgfplotsset{every axis/.append style={
			title style={font=\Large},
			label style={font=\Large},
			tick label style={font=\Large},
			legend style={font=\large},
		}}

% algorithmicx
\algrenewcommand{\algorithmicrequire}{\textbf{Input:}}
\algrenewcommand{\algorithmicensure}{\textbf{Output:}}
\algrenewcommand{\algorithmicwhile}{\textbf{While}}
\algrenewcommand{\algorithmicend}{\textbf{End}}
\algrenewcommand{\algorithmicrepeat}{\textbf{Repeat}}
\algrenewcommand{\algorithmicuntil}{\textbf{Until}}
\algrenewcommand{\algorithmicfor}{\textbf{For}}
\algrenewcommand{\algorithmicif}{\textbf{If}}
\algrenewcommand{\algorithmicelse}{\textbf{Else}}
\algrenewcommand{\algorithmicdo}{}
\algrenewcommand{\algorithmicthen}{}
\algnewcommand{\Initialize}[1]{%
	\State \textbf{Initialize }{#1}
}

% glossaries-extra
\glsdisablehyper
\setabbreviationstyle[acronym]{long-short}
\newacronym{ao}{AO}{Alternating Optimization}
\newacronym{bd}{BD}{Beyond Diagonal}
\newacronym{bcd}{BCD}{Block Coordinate Descent}
\newacronym{d}{D}{Diagonal}
\newacronym{dof}{DoF}{Degrees of Freedom}
\newacronym{siso}{SISO}{Single-Input Single-Output}
\newacronym{miso}{MISO}{Multiple-Input Single-Output}
\newacronym{mimo}{MIMO}{Multiple-Input Multiple-Output}
\newacronym{rcg}{RCG}{Riemannian Conjugate Gradient}
\newacronym{ris}{RIS}{Reconfigurable Intelligent Surface}
\newacronym{gp}{GP}{Globally Passive}
\newacronym{pc}{PC}{Point-to-Point Channel}
\newacronym{p2p}{P2P}{Point-to-Point}
\newacronym{ic}{IC}{Interference Channel}
\newacronym{qos}{QoS}{Quality of Service}
\newacronym{wsr}{WSR}{Weighted Sum-Rate}
\newacronym{snr}{SNR}{Signal-to-Noise Ratio}
\newacronym{svd}{SVD}{Singular Value Decomposition}
\newacronym{mmse}{MMSE}{Minimum Mean-Square Error}
\newacronym{wmmse}{WMMSE}{Weighted \gls{mmse}}
\newacronym{mse}{MSE}{Mean-Square Error}
\newacronym{los}{LoS}{Line-of-Sight}
\newacronym{csi}{CSI}{Channel State Information}
\newacronym{cscg}{CSCG}{Circularly Symmetric Complex Gaussian}
\newacronym{sca}{SCA}{Successive Convex Approximation}
\newacronym{kkt}{KKT}{Karush-Kuhn-Tucker}
\newacronym{rf}{RF}{Radio Frequency}

% * page limit hacks
\captionsetup[subfloat]{farskip=1pt,captionskip=1pt}
\setlength{\abovecaptionskip}{2pt plus 2pt minus 2pt}

% \setlist{nosep}

% \usepackage{xpatch}
\setlength{\topsep}{0pt}
\setlength{\partopsep}{0pt plus 0pt minus 0pt}
\setlength{\parskip}{0pt}
% \setlength{\parindent}{0pt}
% \makeatletter
% \def\thm@space@setup{%
% 	\thm@preskip=1mm
% 	\thm@postskip=\thm@preskip % or whatever, if you don't want them to be equal
% }
% \xpatchcmd{\proof}{\topsep6\p@\@plus6\p@\relax}{}{}{}
% \makeatother

% \usepackage{titlesec}
% \def\subparagraph{}
% \titlespacing*{\section}{0pt}{0pt}{0pt}
% \titlespacing{\subsection}{0pt}{0pt}{0pt}
% \renewcommand{\thesubsubsection}{\arabic{subsubsection}}
% \titleformat{\subsubsection}[runin]{\itshape}{\thesubsubsection)}{1ex}{}
% \titlespacing*{\subsubsection}{0pt}{0pt}{*1}

\usepackage{setspace}
\abovedisplayskip=1mm
\belowdisplayskip=1mm
\abovedisplayshortskip=1mm
\belowdisplayshortskip=1mm
\setlength{\jot}{0.1mm}
\setlength{\floatsep}{1mm}
\setlength{\textfloatsep}{1mm}
\setlength{\intextsep}{1mm}
\setlength{\skip\footins}{1mm}
% \addtolength{\parindent}{-1mm}

% * legacy
% % Double superscript
% \newcommand\herm[2][-4]{{#2}^{\mkern#1mu\mathsf{H}}}

% % algpseudocode
% \makeatletter
% \renewcommand{\fnum@algorithm}{\fname@algorithm{} \thealgorithm:}
% \newcommand\setalgorithmcaptionfont[1]{%
% 	\let\my@floatc@ruled\floatc@ruled          % save \floatc@ruled
% 	\def\floatc@ruled{%
% 		\global\let\floatc@ruled\my@floatc@ruled % restore \floatc@ruled
% 		#1\floatc@ruled}}
% \makeatother

\externaldocument{main}

\makeatletter
\def\bibcite#1#2{%
	\@newl@bel{b}{#1\@extra@binfo}{%
		\hyper@@link[cite]{}{cite.#1\@extra@b@citeb}{\romannumeral 0#2}%
	}%
}
\renewcommand*{\@biblabel}[1]{[\romannumeral 0#1]}
\renewcommand{\theequation}{\roman{equation}}
\makeatother




\begin{document}

\begin{appendix}
	\begin{subsection}{Proof of Proposition \ref{pp:dof}}
		\label{ap:dof}
		It suffices to consider the rank of the indirect channel.
		Denote the \gls{svd} of the backward and forward channels as
		\begin{equation*}
			\mathbf{H}_\mathrm{B/F} = \begin{bmatrix}
				\mathbf{U}_{\mathrm{B/F},1} & \mathbf{U}_{\mathrm{B/F},2}
			\end{bmatrix}
			\begin{bmatrix}
				\mathbf{\Sigma}_{\mathrm{B/F},1} & \mathbf{0} \\
				\mathbf{0}                     & \mathbf{0}
			\end{bmatrix}
			\begin{bmatrix}
				\mathbf{V}_{\mathrm{B/F},1}^\mathsf{H} \\
				\mathbf{V}_{\mathrm{B/F},2}^\mathsf{H}
			\end{bmatrix},
		\end{equation*}
		where $\mathbf{U}_{\mathrm{B/F},1}$ and $\mathbf{V}_{\mathrm{B/F},1}$ are any left and right singular matrices of $\mathbf{H}_\mathrm{B/F}$ corresponding to non-zero singular values $\mathbf{\Sigma}_{\mathrm{B/F},1}$, and $\mathbf{U}_{\mathrm{B/F},2}$ and $\mathbf{V}_{\mathrm{B/F},2}$ are those corresponding to zero singular values.
		The rank of the indirect channel is \cite[(16.5.10.b)]{Hogben2013}
		\begin{equation*}
			\begin{split}
				\rank(\mathbf{H}_\mathrm{B} \mathbf{\Theta} \mathbf{H}_\mathrm{F})
				& = \rank(\mathbf{H}_\mathrm{B}) - \dim \bigl(\ker(\mathbf{H}_\mathrm{F}^\mathsf{H} \mathbf{\Theta}^\mathsf{H}) \cap \ran(\mathbf{H}_\mathrm{B}^\mathsf{H})\bigr) \\
				& = \rank(\mathbf{H}_\mathrm{B}) - \dim \bigl(\ran(\mathbf{\Theta} \mathbf{U}_{\mathrm{F},2}) \cap \ran(\mathbf{V}_{\mathrm{B},1})\bigr) \\
				& \triangleq r_\mathrm{B} - r_\mathrm{L}(\mathbf{\Theta}),
			\end{split}
		\end{equation*}
		where we define $r_\mathrm{L}(\mathbf{\Theta}) \triangleq \dim \bigl(\ran(\mathbf{\Theta} \mathbf{U}_{\mathrm{F},2}) \cap \ran(\mathbf{V}_{\mathrm{B},1})\bigr)$ and $r_\mathrm{B/F} \triangleq \rank(\mathbf{H}_\mathrm{B/F})$.
		Since $\mathbf{U}_{\mathrm{F},2} \in \mathbb{U}^{N_\mathrm{S} \times (N_\mathrm{S} - r_\mathrm{F})}$ and $\mathbf{V}_{\mathrm{B},1} \in \mathbb{U}^{N_\mathrm{S} \times r_\mathrm{B}}$, we have $\max(r_\mathrm{B} - r_\mathrm{F}, 0) \le r_\mathrm{L}(\mathbf{\Theta}) \le \min(N_\mathrm{S} - r_\mathrm{F}, r_\mathrm{B})$
		and thus
		\begin{equation}
			\label{iq:rank_indirect}
			\max(r_\mathrm{B} + r_\mathrm{F} - N_\mathrm{S}, 0) \le \rank(\mathbf{H}_\mathrm{B} \mathbf{\Theta} \mathbf{H}_\mathrm{F}) \le \min(r_\mathrm{B}, r_\mathrm{F}).
		\end{equation}
		To attain the upper bound in \eqref{iq:rank_indirect}, the \gls{ris} needs to minimize $r_\mathrm{L}(\mathbf{\Theta})$ by aligning the ranges of $\mathbf{\Theta} \mathbf{U}_{\mathrm{F},2}$ and $\mathbf{V}_{\mathrm{B},2}$ as much as possible.
		This is achieved by
		\begin{equation}
			\label{eq:ris_dof_max}
			\mathbf{\Theta}_{\textnormal{DoF-max}}^{\textnormal{MIMO}} = \mathbf{Q}_{\mathrm{B},2} \mathbf{Q}_{\mathrm{F},2}^\mathsf{H},
		\end{equation}
		where $\mathbf{Q}_{\mathrm{B},2}$ and $\mathbf{Q}_{\mathrm{F},2}$ are the unitary matrices of the QR decomposition of $\mathbf{V}_{\mathrm{B},2}$ and $\mathbf{U}_{\mathrm{F},2}$, respectively.
		Similarly, the lower bound in \eqref{iq:rank_indirect} is attained at
		\begin{equation}
			\label{eq:ris_dof_min}
			\mathbf{\Theta}_{\textnormal{DoF-min}}^{\textnormal{MIMO}} = \mathbf{Q}_{\mathrm{B},1} \mathbf{Q}_{\mathrm{F},2}^\mathsf{H},
		\end{equation}
		where $\mathbf{Q}_{\mathrm{B},1}$ is the unitary matrix of the QR decomposition of $\mathbf{V}_{\mathrm{B},1}$.
		While the \gls{dof}-optimal structures \eqref{eq:ris_dof_max} and \eqref{eq:ris_dof_min} are always feasible for fully-connected \gls{bd}-\gls{ris}, they are generally infeasible for \gls{d}-\gls{ris} unless there exist some QR decomposition that diagonalize $\mathbf{Q}_{\mathrm{B},2} \mathbf{Q}_{\mathrm{F},2}^\mathsf{H}$ and $\mathbf{Q}_{\mathrm{B},1} \mathbf{Q}_{\mathrm{F},2}^\mathsf{H}$ simultaneously.
		That is, \gls{bd}-\gls{ris} may achieve a larger or smaller number of \gls{dof} of indirect channel, and thus equivalent channel, than \gls{d}-\gls{ris}.
	\end{subsection}

	\begin{subsection}{Proof of Proposition \ref{pp:rd}}
		\label{ap:rank_deficient}
		We consider rank-$k$ forward channel and the proof follows similarly for rank-$k$ backward channel.
		Let $\mathbf{H}_\mathrm{F} = \mathbf{U}_\mathrm{F} \mathbf{\Sigma}_\mathrm{F} \mathbf{V}_\mathrm{F}^\mathsf{H}$ be the \gls{svd} of the forward channel.
		The channel Gram matrix $\mathbf{G} \triangleq \mathbf{H} \mathbf{H}^\mathsf{H} $ can be written as
		\begin{equation*}
			\begin{split}
				\mathbf{G}
				& = \mathbf{H}_\mathrm{D} \mathbf{H}_\mathrm{D}^\mathsf{H} + \mathbf{H}_\mathrm{B} \mathbf{\Theta} \mathbf{U}_\mathrm{F} \mathbf{\Sigma}_\mathrm{F} \mathbf{\Sigma}_\mathrm{F}^\mathsf{H} \mathbf{U}_\mathrm{F}^\mathsf{H} \mathbf{\Theta}^\mathsf{H} \mathbf{H}_\mathrm{B}^\mathsf{H}                                                         \\
				& \quad + \mathbf{H}_\mathrm{B} \mathbf{\Theta} \mathbf{U}_\mathrm{F} \mathbf{\Sigma}_\mathrm{F} \mathbf{V}_\mathrm{F}^\mathsf{H} \mathbf{H}_\mathrm{D}^\mathsf{H} + \mathbf{H}_\mathrm{D} \mathbf{V}_\mathrm{F} \mathbf{\Sigma}_\mathrm{F} \mathbf{U}_\mathrm{F}^\mathsf{H} \mathbf{\Theta}^\mathsf{H} \mathbf{H}_\mathrm{B}^\mathsf{H}       \\
				& = \mathbf{H}_\mathrm{D} (\mathbf{I} - \mathbf{V}_\mathrm{F} \mathbf{V}_\mathrm{F}^\mathsf{H}) \mathbf{H}_\mathrm{D}^\mathsf{H}                                                                                                                                                                                                               \\
				& \quad + (\mathbf{H}_\mathrm{B} \mathbf{\Theta} \mathbf{U}_\mathrm{F} \mathbf{\Sigma}_\mathrm{F} + \mathbf{H}_\mathrm{D} \mathbf{V}_\mathrm{F}) (\mathbf{\Sigma}_\mathrm{F} \mathbf{U}_\mathrm{F}^\mathsf{H} \mathbf{\Theta}^\mathsf{H} \mathbf{H}_\mathrm{B}^\mathsf{H} + \mathbf{V}_\mathrm{F}^\mathsf{H} \mathbf{H}_\mathrm{D}^\mathsf{H}) \\
				& = \mathbf{Y} + \mathbf{Z} \mathbf{Z}^\mathsf{H},
			\end{split}
		\end{equation*}
		where we define $\mathbf{Y} \triangleq \mathbf{H}_\mathrm{D} (\mathbf{I} - \mathbf{V}_\mathrm{F} \mathbf{V}_\mathrm{F}^\mathsf{H}) \mathbf{H}_\mathrm{D}^\mathsf{H} \in \mathbb{H}^{N_\mathrm{R} \times N_\mathrm{R}}$ and $\mathbf{Z} \triangleq \mathbf{H}_\mathrm{B} \mathbf{\Theta} \mathbf{U}_\mathrm{F} \mathbf{\Sigma}_\mathrm{F} + \mathbf{H}_\mathrm{D} \mathbf{V}_\mathrm{F} \in \mathbb{C}^{N_\mathrm{R} \times k}$.
		That is to say, $\mathbf{G}$ can be expressed as a Hermitian matrix plus $k$ rank-1 perturbations.
		According to the Cauchy interlacing formula \cite[Theorem 8.4.3]{Golub2013}, the $n$-th eigenvalue of $\mathbf{G}$ is bounded by
		\begin{align}
			\lambda_n(\mathbf{G}) & \le \lambda_{n-k}(\mathbf{Y}), &  & \text{if } n > k, \label{iq:ev_rd_max}          \\
			\lambda_n(\mathbf{G}) & \ge \lambda_n(\mathbf{Y}),     &  & \text{if } n < N - k + 1 \label{iq:ev_rd_min}.
		\end{align}
		Since $\mathbf{Y} = \mathbf{T} \mathbf{T}^\mathsf{H}$ is positive semi-definite, taking the square roots of \eqref{iq:ev_rd_max} and \eqref{iq:ev_rd_min} gives \eqref{iq:sv_rd_max} and \eqref{iq:sv_rd_min}.
	\end{subsection}

	\begin{subsection}{Proof of Proposition \ref{pp:nd}}
		\label{ap:nd}
		Let $\mathbf{H}_\mathrm{B} = \mathbf{U}_\mathrm{B} \mathbf{\Sigma}_\mathrm{B} \mathbf{V}_\mathrm{B}^\mathsf{H}$ and $\mathbf{H}_\mathrm{F} = \mathbf{U}_\mathrm{F} \mathbf{\Sigma}_\mathrm{F} \mathbf{V}_\mathrm{F}^\mathsf{H}$ be the \gls{svd} of the backward and forward channels, respectively.
		The scattering matrix of fully-connected \gls{bd}-\gls{ris} can be decomposed as
		\begin{equation}
			\label{eq:ris_decompose}
			\mathbf{\Theta} = \mathbf{V}_\mathrm{B} \mathbf{X} \mathbf{U}_\mathrm{F}^\mathsf{H},
		\end{equation}
		where $\mathbf{X} \in \mathbb{U}^{N_\mathrm{S} \times N_\mathrm{S}}$ is a unitary matrix to be designed.
		The equivalent channel is thus a function of $\mathbf{X}$
		\begin{equation}
			\label{eq:channel_nd}
			\mathbf{H} = \mathbf{H}_\mathrm{B} \mathbf{\Theta} \mathbf{H}_\mathrm{F} = \mathbf{U}_\mathrm{B} \mathbf{\Sigma}_\mathrm{B} \mathbf{X} \mathbf{\Sigma}_\mathrm{F} \mathbf{V}_\mathrm{F}^\mathsf{H}.
		\end{equation}
		Since $\sv(\mathbf{U} \mathbf{A} \mathbf{V}^\mathsf{H}) = \sv(\mathbf{A})$ for unitary $\mathbf{U}$ and $\mathbf{V}$, we have
		\begin{equation}
			\label{eq:sv_factor}
			\begin{split}
				\sv(\mathbf{H}) & = \sv(\mathbf{U}_\mathrm{B} \mathbf{\Sigma}_\mathrm{B} \mathbf{X} \mathbf{\Sigma}_\mathrm{F} \mathbf{V}_\mathrm{F}^\mathsf{H})                                                                     \\
								& = \sv(\mathbf{\Sigma}_\mathrm{B} \mathbf{X} \mathbf{\Sigma}_\mathrm{F})                                                                                                                            \\
								& = \sv(\bar{\mathbf{U}}_\mathrm{B} \mathbf{\Sigma}_\mathrm{B} \mathbf{\bar{V}}_\mathrm{B}^\mathsf{H} \bar{\mathbf{U}}_\mathrm{F} \mathbf{\Sigma}_\mathrm{F} \mathbf{\bar{V}}_\mathrm{F}^\mathsf{H}) \\
								& = \sv(\mathbf{BF}),
			\end{split}
		\end{equation}
		where $\bar{\mathbf{U}}_{\mathrm{B}} \in \mathbb{U}^{N_\mathrm{R} \times N_\mathrm{R}}$, $\bar{\mathbf{V}}_\mathrm{B}, \bar{\mathbf{U}}_\mathrm{F} \in \mathbb{U}^{N_\mathrm{S} \times N_\mathrm{S}}$, and $\bar{\mathbf{V}}_\mathrm{F} \in \mathbb{U}^{N_\mathrm{T} \times N_\mathrm{T}}$ can be designed arbitrarily.
	\end{subsection}

	\begin{subsection}{Proof of Corollary \ref{co:nd_sv_prod_tail}}
		\label{ap:nd_sv_prod_tail}
		\eqref{iq:sv_nd_prod_largest} follows from \eqref{iq:horn} when $r = k$.
		On the other hand, if we can prove
		\begin{equation}
			\label{eq:sv_prod_all_ext}
			\prod_{n=1}^{\bar{N}} \sigma_n(\mathbf{H}) = \prod_{n=1}^{\bar{N}} \sigma_n(\mathbf{H}_\mathrm{B}) \sigma_n(\mathbf{H}_\mathrm{F}),
		\end{equation}
		then \eqref{iq:sv_nd_prod_smallest} follows from \eqref{iq:sv_nd_prod_largest} and the non-negativity of singular values.
		To see \eqref{eq:sv_prod_all_ext}, we start from a stricter result
		\begin{equation}
			\label{eq:sv_product_all}
			\prod_{n=1}^{N_\mathrm{S}} \sigma_n(\mathbf{H}) = \prod_{n=1}^{N_\mathrm{S}} \sigma_n(\mathbf{H}_\mathrm{B}) \sigma_n(\mathbf{H}_\mathrm{F}),
		\end{equation}
		which is provable by cases.
		When $N_\mathrm{S} > N$, both sides of \eqref{eq:sv_product_all} become zero since $\sigma_n(\mathbf{H}) = \sigma_n(\mathbf{H}_\mathrm{B}) = \sigma_n(\mathbf{H}_\mathrm{F}) = 0$ for $n > N$.
		When $N_\mathrm{S} \le N$, we have
		\begin{equation*}
			\begin{split}
				\prod\nolimits_{n=1}^{N_\mathrm{S}} \sigma_n(\mathbf{H})
				& = \prod\nolimits_{n=1}^{N_\mathrm{S}} \sigma_n(\mathbf{\Sigma}_\mathrm{B} \mathbf{X} \mathbf{\Sigma}_\mathrm{F})             \\
				& = \prod\nolimits_{n=1}^{N_\mathrm{S}} \sigma_n(\hat{\mathbf{\Sigma}}_\mathrm{B} \mathbf{X} \hat{\mathbf{\Sigma}}_\mathrm{F}) \\
				& = \det\bigl(\hat{\mathbf{\Sigma}}_\mathrm{B} \mathbf{X} \hat{\mathbf{\Sigma}}_\mathrm{F}\bigr)                               \\
				& = \det\bigl(\hat{\mathbf{\Sigma}}_\mathrm{B}\bigr) \det(\mathbf{X}) \det\bigl(\hat{\mathbf{\Sigma}}_\mathrm{F}\bigr)         \\
				& = \prod\nolimits_{n=1}^{N_\mathrm{S}} \sigma_n(\mathbf{\Sigma}_\mathrm{B}) \sigma_n(\mathbf{\Sigma}_\mathrm{F}),
			\end{split}
		\end{equation*}
		where the first equality follows from \eqref{eq:sv_factor} and $\hat{\mathbf{\Sigma}}_\mathrm{B}, \hat{\mathbf{\Sigma}}_\mathrm{F}$ truncate $\mathbf{\Sigma}_\mathrm{B}, \mathbf{\Sigma}_\mathrm{F}$ to square matrices of dimension $N_\mathrm{S}$, respectively.
		It is evident that \eqref{eq:sv_product_all} implies \eqref{eq:sv_prod_all_ext} and thus \eqref{iq:sv_nd_prod_smallest}.
	\end{subsection}

	\begin{subsection}{Proof of Corollary \ref{co:nd_sv_indl}}
		\label{ap:nd_sv_indl}
		In \eqref{iq:sv_nd_indl}, the set of upper bounds
		\begin{equation}
			\label{iq:sv_nd_indl_set_max}
			\bigl\{\sigma_n(\mathbf{H}) \le \sigma_i(\mathbf{H}_\mathrm{B}) \sigma_j(\mathbf{H}_\mathrm{F}) \mid [i,j,k] \in [N_\mathrm{S}]^3, i+j=n+1\bigr\}
		\end{equation}
		is a special case of \eqref{iq:horn} with $(I, J, K) \in [N_\mathrm{S}]^3$.
		The minimum\footnote{One may think to take the maximum of those upper bounds as the problem of interest is the attainable dynamic range of $n$-th singular value. This is infeasible since the singular values will be reordered.} of \eqref{iq:sv_nd_indl_set_max} is selected as the tightest upper bound in \eqref{iq:sv_nd_indl}.
		On the other hand, the set of lower bounds
		\begin{equation}
			\label{iq:sv_nd_indl_set_min}
			\bigl\{\sigma_n(\mathbf{H}) \ge \sigma_i(\mathbf{H}_\mathrm{B}) \sigma_j(\mathbf{H}_\mathrm{F}) \mid [i,j,k] \in [N_\mathrm{S}]^3, i+j=n+N_\mathrm{S}\bigr\}
		\end{equation}
		can be induced by \eqref{iq:sv_nd_indl_set_max}, \eqref{eq:sv_product_all}, and the non-negativity of singular values.
		The maximum of \eqref{iq:sv_nd_indl_set_min} is selected as the tightest lower bound in \eqref{iq:sv_nd_indl}.
		Interested readers are also referred to \cite[(2.0.3)]{Zhang2005}.

		To attain the upper bound, the \gls{bd}-\gls{ris} needs to maximize the minimum of the first $n$ channel singular values.
		It follows from \eqref{eq:ris_nd_sv_indl_max} that
		\begin{equation*}
			\begin{split}
				\sv(\mathbf{H})
				& = \sv(\mathbf{H}_\mathrm{B} \mathbf{V}_\mathrm{B} \mathbf{P} \mathbf{U}_\mathrm{F}^\mathsf{H} \mathbf{H}_\mathrm{F})                                                                                                                         \\
				& = \sv(\mathbf{U}_\mathrm{B} \mathbf{\Sigma}_\mathrm{B} \mathbf{V}_\mathrm{B}^\mathsf{H} \mathbf{V}_\mathrm{B} \mathbf{P} \mathbf{U}_\mathrm{F}^\mathsf{H} \mathbf{U}_\mathrm{F} \mathbf{\Sigma}_\mathrm{F} \mathbf{U}_\mathrm{F}^\mathsf{H}) \\
				& = \sv(\mathbf{\Sigma}_\mathrm{B} \mathbf{P} \mathbf{\Sigma}_\mathrm{F}).
			\end{split}
		\end{equation*}
		On the one hand, $\mathbf{P}_{ij}=1$ with $(i, j)$ satisfying \eqref{eq:idx_nd_sv_indl_max} ensures $\min_{i+j=n+1} \sigma_i(\mathbf{H}_\mathrm{B}) \sigma_j(\mathbf{H}_\mathrm{F})$ is a singular value of $\mathbf{H}$.
		It is actually among the first $n$ since the number of pairs $(i',j')$ not majorized by $(i,j)$ is $n-1$.
		On the other hand, \eqref{eq:perm_nd_sv_indl_max} ensures the first $(n-1)$-th singular values are no smaller than $\min_{i+j=n+1} \sigma_i(\mathbf{H}_\mathrm{B}) \sigma_j(\mathbf{H}_\mathrm{F})$.
		Combining both facts, we claim the upper bound $\sigma_n(\mathbf{H}) = \min_{i+j=n+1} \sigma_i(\mathbf{H}_\mathrm{B}) \sigma_j(\mathbf{H}_\mathrm{F})$ is attainable by \eqref{eq:ris_nd_sv_indl_max}.
		The attainability of the lower bound can be proved similarly and the details are omitted.
	\end{subsection}

	\begin{subsection}{Proof of Corollary \ref{co:nd_power}}
		\label{ap:nd_power}
		From \eqref{eq:ris_decompose} and \eqref{eq:channel_nd} we have
		\begin{equation}
			\begin{split}
				\lVert \mathbf{H} \rVert _\mathrm{F}^2
				& = \tr \bigl(\mathbf{V}_\mathrm{F} \mathbf{\Sigma}_\mathrm{F}^\mathsf{H} \mathbf{X}^\mathsf{H} \mathbf{\Sigma}_\mathrm{B}^\mathsf{H} \mathbf{U}_\mathrm{B}^\mathsf{H} \mathbf{U}_\mathrm{B} \mathbf{\Sigma}_\mathrm{B} \mathbf{X} \mathbf{\Sigma}_\mathrm{F} \mathbf{V}_\mathrm{F}^\mathsf{H}\bigr) \\
				& = \tr \bigl(\mathbf{\Sigma}_\mathrm{B}^\mathsf{H} \mathbf{\Sigma}_\mathrm{B} \cdot \mathbf{X} \mathbf{\Sigma}_\mathrm{F} \mathbf{\Sigma}_\mathrm{F}^\mathsf{H} \mathbf{X}^\mathsf{H}\bigr)                                                                                                         \\
				& \triangleq \tr \bigl(\tilde{\mathbf{B}} \tilde{\mathbf{F}}\bigr),
			\end{split}
		\end{equation}
		where $\mathbf{X} \triangleq \mathbf{V}_\mathrm{B}^\mathsf{H} \mathbf{\Theta} \mathbf{U}_\mathrm{F} \in \mathbb{U}^{N_\mathrm{S} \times N_\mathrm{S}}$, $\tilde{\mathbf{B}} \triangleq \mathbf{\Sigma}_\mathrm{B}^\mathsf{H} \mathbf{\Sigma}_\mathrm{B} \in \mathbb{H}_+^{N_\mathrm{S} \times N_\mathrm{S}}$, and $\tilde{\mathbf{F}} \triangleq \mathbf{X} \mathbf{\Sigma}_\mathrm{F} \mathbf{\Sigma}_\mathrm{F}^\mathsf{H} \mathbf{X}^\mathsf{H} \in \mathbb{H}_+^{N_\mathrm{S} \times N_\mathrm{S}}$.
		By Ruhe's trace inequality for positive semi-definite matrices \cite[(H.1.g) and (H.1.h)]{Marshall2010},
		\begin{equation*}
			\sum_{n=1}^N \lambda_n(\tilde{\mathbf{B}}) \lambda_{N_\mathrm{S}-n+1}(\tilde{\mathbf{F}}) \le \tr \bigl(\tilde{\mathbf{B}} \tilde{\mathbf{F}}\bigr) \le \sum_{n=1}^N \lambda_n(\tilde{\mathbf{B}}) \lambda_n(\tilde{\mathbf{F}}),
		\end{equation*}
		which simplifies to \eqref{iq:power_nd}.
		The upper bound is attained when $\mathbf{X}$ is chosen to match the singular values of $\tilde{\mathbf{F}}$ to those of $\tilde{\mathbf{B}}$ in similar order.
		Apparently this occurs at $\mathbf{X} = \mathbf{I}$ and $\mathbf{\Theta} = \mathbf{V}_\mathrm{B} \mathbf{U}_\mathrm{F}^\mathsf{H}$.
		On the other hand, the lower bound is attained when the singular values of $\tilde{\mathbf{F}}$ and $\tilde{\mathbf{B}}$ are matched in reverse order, namely $\mathbf{X} = \mathbf{J}$ and $\mathbf{\Theta} = \mathbf{V}_\mathrm{B} \mathbf{J} \mathbf{U}_\mathrm{F}^\mathsf{H}$.
	\end{subsection}

	\begin{subsection}{Proof of Corollary \ref{co:nd_capacity_snr_extreme}}
		\label{ap:nd_capacity}
		When perfect \gls{csi} is available at the transmitter, in the low-\gls{snr} regime, the capacity is achieved by dominant eigenmode transmission \cite[(5.26)]{Clerckx2013}
		\begin{align*}
			C_{\rho_\downarrow}
			& = \log\bigl(1 + \rho \lambda_1(\mathbf{H}^\mathsf{H} \mathbf{H})\bigr)        \\
			& = \log\bigl(1 + \rho \sigma_1^2(\mathbf{H})\bigr)                             \\
			& \approx \rho \sigma_1^2(\mathbf{H})                                           \\
			& \le \rho \sigma_1^2(\mathbf{H}_\mathrm{B}) \sigma_1^2(\mathbf{H}_\mathrm{F}),
		\end{align*}
		where the approximation is $\log(1 + x) \approx x$ for small $x$ and the inequality follows from \eqref{iq:sv_nd_prod_largest} with $k=1$.
		In the high-\gls{snr} regime, the capacity is achieved by multiple eigenmode transmission with uniform power location \cite[(5.27)]{Clerckx2013}
		\begin{align*}
			C_{\rho_\uparrow}
			& = \sum\nolimits_{n=1}^N \log\Bigl(1 + \frac{\rho}{N} \lambda_n(\mathbf{H}^\mathsf{H} \mathbf{H})\Bigr)                     \\
			& \approx \sum\nolimits_{n=1}^N \log\Bigl(\frac{\rho}{N} \sigma_n^2(\mathbf{H})\Bigr)                                        \\
			& = N \log \frac{\rho}{N} + \sum\nolimits_{n=1}^N \log \sigma_n^2(\mathbf{H})                                                \\
			& = N \log \frac{\rho}{N} + \log \prod\nolimits_{n=1}^N \sigma_n^2(\mathbf{H})                                               \\
			& \le N \log \frac{\rho}{N} + 2 \log \prod\nolimits_{n=1}^N \sigma_n(\mathbf{H}_\mathrm{B}) \sigma_n(\mathbf{H}_\mathrm{F}),
		\end{align*}
		where the approximation is $\log(1 + x) \approx \log(x)$ for large $x$ and the inequality follows from \eqref{iq:sv_nd_prod_largest} with $k=N$.

		We now show \eqref{eq:ris_nd_rate_max} can achieve the upper bounds in \eqref{iq:capacity_nd_snr_low} and \eqref{iq:capacity_nd_snr_high} simultaneously.
		On the one hand, \eqref{eq:ris_nd_rate_max} is a special case of \eqref{eq:ris_nd_sv_indl_max} with $\mathbf{P} = \mathbf{I}$, which satisfies \eqref{eq:idx_nd_sv_indl_max} and \eqref{eq:perm_nd_sv_indl_max} for $n=1$ and thus attain $\sigma_1(\mathbf{H}) = \sigma_1(\mathbf{H}_\mathrm{B}) \sigma_1(\mathbf{H}_\mathrm{F})$.
		On the other hand, since $\log(\cdot)$ is a monotonic function, we can prove similar to Appendix \ref{ap:nd_power} that $\sum_{n=1}^N \log \sigma_n^2(\mathbf{H}) \le \sum_{n=1}^N \log \sigma_n^2(\mathbf{H}_\mathrm{B}) \sigma_n^2(\mathbf{H}_\mathrm{F})$ and the bound is tight at \eqref{eq:ris_nd_rate_max}.
		The proof is complete.
	\end{subsection}

	\begin{subsection}{Proof of Proposition \ref{pp:shaping}}
		\label{ap:shaping}
		The sub-differential of a symmetric gauge function of singular values of a matrix with respect to the matrix itself is given by \cite[Theorem 2]{Watson1992}
		\begin{equation}
			\partial_{\mathbf{H}^*} f\bigl(\sv(\mathbf{H})\bigr) = \conv \bigl\{ \mathbf{U} \mathbf{D} \mathbf{V}^\mathsf{H} \bigr\},
		\end{equation}
		where $\mathbf{D} \in \mathbb{C}^{N_\mathrm{R} \times N_\mathrm{T}}$ is a rectangular diagonal matrix with $[\mathbf{D}]_{n,n} \in \partial_{\sigma_n(\mathbf{H})} f\bigl(\sv(\mathbf{H})\bigr)$, $\forall n \in [N]$, and $\mathbf{U}$, $\mathbf{V}$ are any left and right singular matrices of $\mathbf{H}$.
		It implies
		\begin{align*}
			\partial f\bigl(\sv(\mathbf{H})\bigr)
			& \ni \tr \bigl(\mathbf{V}^* \mathbf{D}^\mathsf{T} \mathbf{U}^\mathsf{T} \partial \mathbf{H}^*\bigr) \\
			& = \tr \bigl(\mathbf{V}^* \mathbf{D}^\mathsf{T} \mathbf{U}^\mathsf{T} \mathbf{H}_{\mathrm{B},g}^* \partial {\mathbf{\Theta}_g^*} \mathbf{H}_{\mathrm{F},g}^*\bigr) \\
			& = \tr \bigl(\mathbf{H}_{\mathrm{F},g}^* \mathbf{V}^* \mathbf{D}^\mathsf{T} \mathbf{U}^\mathsf{T} \mathbf{H}_{\mathrm{B},g}^* \partial {\mathbf{\Theta}_g^*} \bigr),
		\end{align*}
		and therefore $\mathbf{H}_{\mathrm{B},g}^\mathsf{H} \mathbf{U} \mathbf{D} \mathbf{V}^\mathsf{H} \mathbf{H}_{\mathrm{F},g}^\mathsf{H}$ constitutes a sub-gradient of $f\bigl(\sv(\mathbf{H})\bigr)$ with respect to $\mathbf{\Theta}_g$.
		The convex hull of those sub-gradients is the sub-differential \eqref{eq:shaping_subdiff}.
	\end{subsection}

	\begin{subsection}{Proof of Lemma \ref{lm:rate}}
		\label{ap:rate}
		The differential of $R$ with respect to $\mathbf{\Theta}_g^*$ is \cite{Hjorungnes2007}
		\begin{align*}
			\partial R
			& = \frac{1}{\eta} \tr \biggl\{ \partial \mathbf{H}^* \cdot \mathbf{Q}^\mathsf{T} \mathbf{H}^\mathsf{T} \Bigl(\mathbf{I} + \frac{\mathbf{H}^* \mathbf{Q}^\mathsf{T} \mathbf{H}^\mathsf{T}}{\eta}\Bigr)^{-1} \biggr\}                                                                      \\
			& = \frac{1}{\eta} \tr \biggl\{ \mathbf{H}_{\mathrm{B},g}^* \cdot \partial \mathbf{\Theta}_g^* \cdot \mathbf{H}_{\mathrm{F},g}^* \mathbf{Q}^\mathsf{T} \mathbf{H}^\mathsf{T} \Bigl(\mathbf{I} + \frac{\mathbf{H}^* \mathbf{Q}^\mathsf{T} \mathbf{H}^\mathsf{T}}{\eta}\Bigr)^{-1} \biggr\} \\
			& = \frac{1}{\eta} \tr \biggl\{ \mathbf{H}_{\mathrm{F},g}^* \mathbf{Q}^\mathsf{T} \mathbf{H}^\mathsf{T} \Bigl(\mathbf{I} + \frac{\mathbf{H}^* \mathbf{Q}^\mathsf{T} \mathbf{H}^\mathsf{T}}{\eta}\Bigr)^{-1} \mathbf{H}_{\mathrm{B},g}^* \cdot \partial \mathbf{\Theta}_g^* \biggr\},
		\end{align*}
		and the corresponding complex derivative is \eqref{eq:gradient_eucl_rate}.
	\end{subsection}

	\begin{subsection}{Proof of Proposition \ref{pp:power}}
		\label{ap:power}
		The differential of \eqref{ob:power} with respect to $\mathbf{\Theta}_g^*$ is
		\begin{align*}
			\partial \lVert \mathbf{H} \rVert _\mathrm{F}^2
			& = \tr\bigl(\mathbf{H}_{\mathrm{B},g}^* \cdot \partial \mathbf{\Theta}_g^* \cdot \mathbf{H}_{\mathrm{F},g}^* (\mathbf{H}_\mathrm{D}^\mathsf{T} + \mathbf{H}_\mathrm{F}^\mathsf{T} \mathbf{\Theta}^\mathsf{T} \mathbf{H}_\mathrm{B}^\mathsf{T})\bigr) \\
			& = \tr\bigl(\mathbf{H}_{\mathrm{F},g}^* (\mathbf{H}_\mathrm{D}^\mathsf{T} + \mathbf{H}_\mathrm{F}^\mathsf{T} \mathbf{\Theta}^\mathsf{T} \mathbf{H}_\mathrm{B}^\mathsf{T}) \mathbf{H}_{\mathrm{B},g}^* \cdot \partial \mathbf{\Theta}_g^*\bigr)
		\end{align*}
		and the corresponding complex derivative is
		\begin{equation}
			\frac{\partial \lVert \mathbf{H} \rVert _\mathrm{F}^2}{\partial \mathbf{\Theta}_g^*} = \mathbf{H}_{\mathrm{B},g}^\mathsf{H} (\mathbf{H}_\mathrm{D} + \mathbf{H}_\mathrm{B} \mathbf{\Theta} \mathbf{H}_\mathrm{F}) \mathbf{H}_{\mathrm{F},g}^\mathsf{H} \triangleq \mathbf{M}_g,
		\end{equation}
		whose \gls{svd} is denoted as $\mathbf{M}_g = \mathbf{U}_g \mathbf{\Sigma}_g \mathbf{V}_g^\mathsf{H}$.
		The quadratic objective \eqref{ob:power} can be successively approximated by its first-order Taylor expansion, resulting in the subproblem
		\begin{maxi!}
			{\scriptstyle{\mathbf{\Theta}}}{\sum_g 2 \Re\bigl\{ \tr(\mathbf{\Theta}_g^\mathsf{H} \mathbf{M}_g) \bigr\}}{\label{op:power_ris_taylor}}{\label{ob:power_ris_taylor}}
			\addConstraint{\mathbf{\Theta}_g^\mathsf{H} \mathbf{\Theta}_g=\mathbf{I}, \quad \forall g,}{}{}
		\end{maxi!}
		whose optimal solution is
		\begin{equation}
			\label{eq:ris_power_taylor}
			\tilde{\mathbf{\Theta}}_g = \mathbf{U}_g \mathbf{V}_g^\mathsf{H}, \quad \forall g.
		\end{equation}
		This is because $\Re \bigl\{\tr(\mathbf{\Theta}_g^\mathsf{H} \mathbf{M}_g)\bigr\} = \Re \bigl\{ \tr(\mathbf{\Sigma}_g \mathbf{V}_g^\mathsf{H} \mathbf{\Theta}_g^\mathsf{H} \mathbf{U}_g) \bigr\} \le \tr(\mathbf{\Sigma}_g)$ and the bound is tight when $\mathbf{V}_g^\mathsf{H} \mathbf{\Theta}_g^\mathsf{H} \mathbf{U}_g = \mathbf{I}$.

		Next, we prove that solving the affine approximation \eqref{op:power_ris_taylor} by \eqref{eq:ris_power_taylor} does not decrease \eqref{ob:power}.
		Since $\tilde{\mathbf{\Theta}} = \diag(\tilde{\mathbf{\Theta}}_1, \ldots, \tilde{\mathbf{\Theta}}_G)$ is optimal for \eqref{op:power_ris_taylor}, we have
		\begin{equation}
			\label{iq:power_ris_taylor}
			\begin{split}
				2 \Re \bigl\{ & \sum\limits_g \tr(\tilde{\mathbf{\Theta}}_g^\mathsf{H} \mathbf{H}_{\mathrm{B},g}^\mathsf{H} \mathbf{H}_\mathrm{D} \mathbf{H}_{\mathrm{F},g}^\mathsf{H}) \\
				& + \sum\limits_{g_1,g_2} \tr(\tilde{\mathbf{\Theta}}_{g_1}^\mathsf{H} \mathbf{H}_{\mathrm{B},g_1}^\mathsf{H} \mathbf{H}_{\mathrm{B},g_2} \mathbf{\Theta}_{g_2} \mathbf{H}_{\mathrm{F},g_2} \mathbf{H}_{\mathrm{F},g_1}^\mathsf{H})\bigr\} \\
				\ge 2 \Re \bigl\{ &\sum\limits_g \tr({\mathbf{\Theta}}_g^\mathsf{H} \mathbf{H}_{\mathrm{B},g}^\mathsf{H} \mathbf{H}_\mathrm{D} \mathbf{H}_{\mathrm{F},g}^\mathsf{H}) \\
				& + \sum\limits_{g_1,g_2} \tr({\mathbf{\Theta}}_{g_1}^\mathsf{H} \mathbf{H}_{\mathrm{B},g_1}^\mathsf{H} \mathbf{H}_{\mathrm{B},g_2} \mathbf{\Theta}_{g_2} \mathbf{H}_{\mathrm{F},g_2} \mathbf{H}_{\mathrm{F},g_1}^\mathsf{H})\bigr\}.
			\end{split}
		\end{equation}
		Besides, $\lVert \sum_g \mathbf{H}_{\mathrm{B},g} \tilde{\mathbf{\Theta}}_g \mathbf{H}_{\mathrm{F},g} - \sum_g \mathbf{H}_{\mathrm{B},g} \mathbf{\Theta}_g \mathbf{H}_{\mathrm{F},g} \rVert _\mathrm{F}^2 \ge 0$ implies
		\begin{equation}
			\label{iq:auxiliary_nonnegative}
			\begin{split}
				& \sum\limits_{g_1,g_2} \tr(\mathbf{H}_{\mathrm{F},g_1}^\mathsf{H} \tilde{\mathbf{\Theta}}_{g_1}^\mathsf{H} \mathbf{H}_{\mathrm{B},g_1}^\mathsf{H} \mathbf{H}_{\mathrm{B},g_2} \tilde{\mathbf{\Theta}}_{g_2} \mathbf{H}_{\mathrm{F},g_2}) \\
				& \quad + \sum\limits_{g_1,g_2} \tr(\mathbf{H}_{\mathrm{F},g_1}^\mathsf{H} {\mathbf{\Theta}}_{g_1}^\mathsf{H} \mathbf{H}_{\mathrm{B},g_1}^\mathsf{H} \mathbf{H}_{\mathrm{B},g_2} {\mathbf{\Theta}}_{g_2} \mathbf{H}_{\mathrm{F},g_2}) \\
				\ge & 2 \Re \bigl\{\sum\limits_{g_1,g_2} \tr(\mathbf{H}_{\mathrm{F},g_1}^\mathsf{H} \tilde{\mathbf{\Theta}}_{g_1}^\mathsf{H} \mathbf{H}_{\mathrm{B},g_1}^\mathsf{H} \mathbf{H}_{\mathrm{B},g_2} \mathbf{\Theta}_{g_2} \mathbf{H}_{\mathrm{F},g_2})\bigr\}.
			\end{split}
		\end{equation}
		Adding \eqref{iq:power_ris_taylor} and \eqref{iq:auxiliary_nonnegative}, we have
		\begin{equation}
			\begin{split}
				& 2 \Re \bigl\{\tr(\tilde{\mathbf{\Theta}}^\mathsf{H} \mathbf{H}_\mathrm{B}^\mathsf{H} \mathbf{H}_\mathrm{D} \mathbf{H}_\mathrm{F}^\mathsf{H}) \bigr\} + \tr(\mathbf{H}_\mathrm{F}^\mathsf{H} \tilde{\mathbf{\Theta}}^\mathsf{H} \mathbf{H}_\mathrm{B}^\mathsf{H} \mathbf{H}_\mathrm{B} \tilde{\mathbf{\Theta}} \mathbf{H}_\mathrm{F}) \\
				\ge & 2 \Re \bigl\{\tr({\mathbf{\Theta}}^\mathsf{H} \mathbf{H}_\mathrm{B}^\mathsf{H} \mathbf{H}_\mathrm{D} \mathbf{H}_\mathrm{F}^\mathsf{H}) \bigr\} + \tr(\mathbf{H}_\mathrm{F}^\mathsf{H} {\mathbf{\Theta}}^\mathsf{H} \mathbf{H}_\mathrm{B}^\mathsf{H} \mathbf{H}_\mathrm{B} {\mathbf{\Theta}} \mathbf{H}_\mathrm{F}),
			\end{split}
		\end{equation}
		which suggests that \eqref{ob:power} is non-decreasing as the solution iterates over \eqref{eq:ris_power_taylor}.
		Since \eqref{ob:power} is also bounded from above, the sequence of objective value converges.

		Finally, we prove that any solution when \eqref{eq:auxiliary_power} converges, denoted by $\mathbf{\Theta}'$, is a stationary point of \eqref{op:power}.
		The \gls{kkt} conditions of \eqref{op:power} and \eqref{op:power_ris_taylor} are equivalent in terms of primal/dual feasibility and complementary slackness, while the stationary conditions are respectively, $\forall g$,
		\begin{gather}
			\mathbf{H}_{\mathrm{B},g}^\mathsf{H} (\mathbf{H}_\mathrm{D} + \mathbf{H}_\mathrm{B} \mathbf{\Theta}^\star \mathbf{H}_\mathrm{F}) \mathbf{H}_{\mathrm{F},g}^\mathsf{H} - \mathbf{\Theta}_g^\star \mathbf{\Lambda}_g^\mathsf{H} = 0,\label{eq:power_ris_optimal}\\
			\mathbf{M}_g - \mathbf{\Theta}_g^\star \mathbf{\Lambda}_g^\mathsf{H} = 0.\label{eq:power_ris_taylor_optimal}
		\end{gather}
		When \eqref{eq:auxiliary_power} converges, $\mathbf{H}_{\mathrm{B},g}^\mathsf{H} (\mathbf{H}_\mathrm{D} + \mathbf{H}_\mathrm{B} \mathbf{\Theta}' \mathbf{H}_\mathrm{F}) \mathbf{H}_{\mathrm{F},g}^\mathsf{H} = \mathbf{H}_{\mathrm{B},g}^\mathsf{H} (\mathbf{H}_\mathrm{D} + \mathbf{H}_\mathrm{B} \mathbf{\Theta}^\star \mathbf{H}_\mathrm{F}) \mathbf{H}_{\mathrm{F},g}^\mathsf{H}$ and \eqref{eq:power_ris_taylor_optimal} reduces to \eqref{eq:power_ris_optimal}.
		The proof is thus completed.
	\end{subsection}

	% \begin{subsection}{Proof of Lemma \ref{lm:wsr}}
	% 	\label{ap:wsr}
	% 	The differential of $f = \sum_{k=1}^K \rho_k R_k$ with respect to $\mathbf{\Theta}_g^*$ is
	% 	\begin{align*}
	% 		\partial f
	% 		& = \sum_{k=1}^K \rho_k \tr \Bigl\{ \mathbf{E}_k \mathbf{W}_k^\mathsf{H} \Bigl( \mathbf{H}_{\mathrm{F},g}^{(k)\mathsf{H}} \partial \mathbf{\Theta}_g^\mathsf{H} \mathbf{H}_{\mathrm{B},g}^{(k)\mathsf{H}} \mathbf{Q}_k^{(-1)} \mathbf{H}^{(kk)} \\
	% 		& \quad + \mathbf{H}^{(kk)\mathsf{H}} \mathbf{Q}_k^{(-1)} \mathbf{H}_{\mathrm{B},g}^{(k)} \partial \mathbf{\Theta}_g \mathbf{H}_{\mathrm{F},g}^{(k)} - \mathbf{H}^{(kk)\mathsf{H}} \mathbf{Q}_k^{(-1)} \\
	% 		& \quad \times \sum_{j \ne k} \Bigl(\mathbf{H}_{\mathrm{B},g}^{(k)} \partial \mathbf{\Theta}_g \mathbf{H}_{\mathrm{F},g}^{(j)} \mathbf{W}_j \mathbf{W}_j^\mathsf{H} \mathbf{H}^{(kj)\mathsf{H}} \\
	% 		& \quad + \mathbf{H}^{(kj)} \mathbf{W}_j \mathbf{W}_j^\mathsf{H} \mathbf{H}_{\mathrm{F},g}^{(j)\mathsf{H}} \partial \mathbf{\Theta}_g^\mathsf{H} \mathbf{H}_{\mathrm{B},g}^{(k)\mathsf{H}} \Bigr) \mathbf{Q}_k^{(-1)} \mathbf{H}^{(kk)} \Bigr) \mathbf{W}_k \Bigr\} \\
	% 		& = \sum_{k=1}^K \rho_k \Bigl( \tr \Bigl\{ \mathbf{H}_{\mathrm{B},g}^{(k)\mathsf{H}} \mathbf{Q}_k^{(-1)} \mathbf{H}^{(kk)} \mathbf{W}_k \mathbf{E}_k \mathbf{W}_k^\mathsf{H} \mathbf{H}_{\mathrm{F},g}^{(k)\mathsf{H}} \partial \mathbf{\Theta}_g^\mathsf{H}\Bigr\} \\
	% 		& \quad + \tr \Bigl\{ \mathbf{H}_{\mathrm{F},g}^{(k)\mathsf{H}} \mathbf{W}_k \mathbf{E}_k \mathbf{W}_k^\mathsf{H} \mathbf{H}^{(kk)\mathsf{H}} \mathbf{Q}_k^{(-1)} \mathbf{H}_{\mathrm{B},g}^{(k)\mathsf{H}} \partial \mathbf{\Theta}_g \Bigr\} \\
	% 		& \quad - \tr \Bigl\{ \sum_{j \ne k} \mathbf{H}_{\mathrm{B},g}^{(k)\mathsf{H}} \mathbf{Q}_k^{(-1)} \mathbf{H}^{(kk)} \mathbf{W}_k \mathbf{E}_k \mathbf{W}_k^\mathsf{H} \mathbf{H}^{(kk)\mathsf{H}} \\
	% 		& \quad \quad \times \mathbf{Q}_k^{(-1)} \mathbf{H}^{(kj)} \mathbf{W}_j \mathbf{W}_j^\mathsf{H} \mathbf{H}_{\mathrm{F},g}^{(j)\mathsf{H}} \partial \mathbf{\Theta}_g^\mathsf{H} \Bigr\} \\
	% 		& \quad - \tr \Bigl\{ \sum_{j \ne k} \mathbf{H}_{\mathrm{F},g}^{(j)\mathsf{H}} \mathbf{W}_j \mathbf{W}_j^\mathsf{H} \mathbf{H}^{(kj)\mathsf{H}} \mathbf{Q}_k^{(-1)} \mathbf{H}^{(kk)} \mathbf{W}_k \\
	% 		& \quad \quad \times \mathbf{E}_k \mathbf{W}_k^\mathsf{H} \mathbf{H}^{(kk)\mathsf{H}} \mathbf{Q}_k^{(-1)} \mathbf{H}_{\mathrm{B},g}^{(k)} \partial \mathbf{\Theta}_g \Bigr\} \Bigr),
	% 	\end{align*}
	% 	and the corresponding complex derivative is \eqref{eq:gradient_eucl_wsr}.
	% \end{subsection}

	% \begin{subsection}{Proof of Proposition \ref{pp:interference}}
	% 	\label{ap:interference}
	% 	Minimizing \eqref{ob:interference} is equivalent to maximizing
	% 	\begin{align*}
	% 		f(\mathbf{\Theta}) & = - I + \sum_g \beta_g^{(k)} \tr \Bigl\{ \mathbf{H}_{\mathrm{F},g}^{(j)\mathsf{H}} \mathbf{H}_{\mathrm{F},g}^{(j)} \Bigr\} \\
	% 		& = \sum_{k=1}^K \sum_{j \ne k} - \tr \Bigl\{ \sum_g \mathbf{H}_{\mathrm{F},g}^{(j)} \mathbf{H}_{\mathrm{D}}^{(kj)\mathsf{H}} \mathbf{H}_{\mathrm{B},g}^{(k)} \mathbf{\Theta}_g\Bigr\} \\
	% 		& \quad - \tr \Bigl\{ \sum_g \mathbf{H}_{\mathrm{B},g}^{(k)\mathsf{H}} \mathbf{H}_{\mathrm{D}}^{(kj)} \mathbf{H}_{\mathrm{F},g}^{(j)\mathsf{H}} \mathbf{\Theta}_g^\mathsf{H} \Bigr\} \\
	% 		& \quad - \tr \Bigl\{ \sum_{g_1=1}^G \sum_{g_2 \ne g_1} \mathbf{H}_{\mathrm{B},g_2}^{(k)\mathsf{H}} \mathbf{H}_{\mathrm{B},g_1}^{(k)} \mathbf{\Theta}_{g_1} \mathbf{H}_{\mathrm{F},g_1}^{(j)} \mathbf{H}_{\mathrm{F},g_2}^{(j)\mathsf{H}} \mathbf{\Theta}_{g_2}^\mathsf{H}\Bigr\} \\
	% 		& \quad - \tr \Bigl\{ \sum_g \mathbf{H}_{\mathrm{B},g}^{(k)\mathsf{H}} \mathbf{H}_{\mathrm{B},g}^{(k)} \mathbf{\Theta}_{g} \mathbf{H}_{\mathrm{F},g}^{(j)\mathsf{H}} \mathbf{H}_{\mathrm{F},g}^{(j)} \mathbf{\Theta}_{g}^\mathsf{H} \Bigr\} \\
	% 		& \quad + \tr \Bigl\{ \sum_g \beta_g^{(k)} \mathbf{\Theta}_{g} \mathbf{H}_{\mathrm{F},g}^{(j)\mathsf{H}} \mathbf{H}_{\mathrm{F},g}^{(j)} \mathbf{\Theta}_{g}^\mathsf{H} \Bigr\} \\
	% 		& = \sum_{k=1}^K \sum_{j \ne k} - \tr \Bigl\{ \sum_g \mathbf{H}_{\mathrm{F},g}^{(j)} \mathbf{H}_{\mathrm{D}}^{(kj)\mathsf{H}} \mathbf{H}_{\mathrm{B},g}^{(k)} \mathbf{\Theta}_g\Bigr\} \\
	% 		& \quad - \tr \Bigl\{ \sum_g \mathbf{H}_{\mathrm{B},g}^{(k)\mathsf{H}} \mathbf{H}_{\mathrm{D}}^{(kj)} \mathbf{H}_{\mathrm{F},g}^{(j)\mathsf{H}} \mathbf{\Theta}_g^\mathsf{H} \Bigr\} \\
	% 		& \quad - \tr \Bigl\{ \sum_{g_1=1}^G \sum_{g_2 \ne g_1} \mathbf{H}_{\mathrm{B},g_2}^{(k)\mathsf{H}} \mathbf{H}_{\mathrm{B},g_1}^{(k)} \mathbf{\Theta}_{g_1} \mathbf{H}_{\mathrm{F},g_1}^{(j)} \mathbf{H}_{\mathrm{F},g_2}^{(j)\mathsf{H}} \mathbf{\Theta}_{g_2}^\mathsf{H}\Bigr\} \\
	% 		& \quad + \tr \Bigl\{ \sum_g \mathbf{B}_{g}^{(k)} \mathbf{\Theta}_{g} \mathbf{H}_{\mathrm{F},g}^{(j)\mathsf{H}} \mathbf{H}_{\mathrm{F},g}^{(j)} \mathbf{\Theta}_{g}^\mathsf{H} \Bigr\},
	% 	\end{align*}
	% 	where $\mathbf{B}_{g}^{(k)} = \beta_g^{(k)} \mathbf{I} - \mathbf{H}_{\mathrm{B},g}^{(k)\mathsf{H}} \mathbf{H}_{\mathrm{B},g}^{(k)}$ and the relaxation constant $\beta_g^{(k)}$ can be chosen arbitrarily.
	% 	One can choose $\beta_g^{(k)} = \lambda_1(\mathbf{H}_{\mathrm{B},g}^{(k)\mathsf{H}} \mathbf{H}_{\mathrm{B},g}^{(k)})$ to ensure the positive semi-definiteness of $\mathbf{B}_{g}^{(k)}$ and formulate a quadratic function to be maximized.
	% 	The remaining proof is similar to Appendix \ref{ap:power} and omitted here.
	% \end{subsection}
\end{appendix}

\begin{section}{Acknowledgement}
	The authors would like to thank the anonymous reviewers for their insightful criticisms and suggestions that helped us correct several technical errors.
\end{section}

\bibliographystyle{IEEEtran}
\bibliography{library.bib}
\end{document}
