\documentclass[
	fontsize=11pt,
	paper=a4,
	foldmarks=false
]{scrartcl}

\usepackage{adjustbox}
\usepackage{algorithm}
\usepackage{algpseudocode}
\usepackage{amsfonts}
\usepackage{amsmath}
\usepackage{amssymb}
\usepackage{amsthm}
\usepackage{booktabs}
\usepackage{cases}
\usepackage{cite}
\usepackage[acronym]{glossaries-extra}
\usepackage{hyperref}
\usepackage{mathtools}
\usepackage[nopatch=footnote]{microtype}
\usepackage[short]{optidef}
\usepackage{pgfplots}
\usepackage[subtle]{savetrees}
\usepackage{siunitx}
\usepackage[caption=false,font=footnotesize,subrefformat=parens,labelformat=parens]{subfig}
\usepackage{tabularx}
\usepackage{tikz}
\usepackage{epstopdf}
\usepackage{multirow}
\usepackage{enumitem}

% amsmath
\allowdisplaybreaks
\DeclareMathOperator{\diag}{diag}
\DeclareMathOperator{\tr}{tr}
\DeclareMathOperator{\sv}{sv}
% \DeclareMathOperator{\card}{card}
\DeclareMathOperator{\conv}{conv}
\DeclareMathOperator{\rank}{rank}
\DeclareMathOperator{\ran}{ran}
% \DeclareMathOperator{\ker}{ker}

% amsthm
\theoremstyle{plain}
\newtheorem{proposition}{Proposition}
\newtheorem{corollary}{Corollary}[proposition]
\newtheorem{lemma}{Lemma}

\theoremstyle{definition}
\newtheorem{definition}{Definition}
\newtheorem{example}{Example}
\newtheorem{remark}{Remark}

% enumitem
\setlist[itemize]{leftmargin=5mm}
\setlist[enumerate]{leftmargin=5mm}

% PGF/TikZ
\usetikzlibrary{arrows,calc,matrix,patterns,plotmarks,positioning,shapes}
\usetikzlibrary{decorations.pathmorphing,decorations.pathreplacing,decorations.shapes,shapes.geometric}
\usepgfplotslibrary{groupplots,patchplots}
\pgfplotsset{compat=1.18}
\pgfplotsset{every axis/.append style={
			title style={font=\Large},
			label style={font=\Large},
			tick label style={font=\Large},
			legend style={font=\large},
		}}

% algorithmicx
\algrenewcommand{\algorithmicrequire}{\textbf{Input:}}
\algrenewcommand{\algorithmicensure}{\textbf{Output:}}
\algrenewcommand{\algorithmicwhile}{\textbf{While}}
\algrenewcommand{\algorithmicend}{\textbf{End}}
\algrenewcommand{\algorithmicrepeat}{\textbf{Repeat}}
\algrenewcommand{\algorithmicuntil}{\textbf{Until}}
\algrenewcommand{\algorithmicfor}{\textbf{For}}
\algrenewcommand{\algorithmicif}{\textbf{If}}
\algrenewcommand{\algorithmicelse}{\textbf{Else}}
\algrenewcommand{\algorithmicdo}{}
\algrenewcommand{\algorithmicthen}{}
\algnewcommand{\Initialize}[1]{%
	\State \textbf{Initialize }{#1}
}

% glossaries-extra
\glsdisablehyper
\setabbreviationstyle[acronym]{long-short}
\newacronym{ao}{AO}{Alternating Optimization}
\newacronym{bd}{BD}{Beyond Diagonal}
\newacronym{bcd}{BCD}{Block Coordinate Descent}
\newacronym{d}{D}{Diagonal}
\newacronym{dof}{DoF}{Degrees of Freedom}
\newacronym{siso}{SISO}{Single-Input Single-Output}
\newacronym{miso}{MISO}{Multiple-Input Single-Output}
\newacronym{mimo}{MIMO}{Multiple-Input Multiple-Output}
\newacronym{rcg}{RCG}{Riemannian Conjugate Gradient}
\newacronym{ris}{RIS}{Reconfigurable Intelligent Surface}
\newacronym{gp}{GP}{Globally Passive}
\newacronym{pc}{PC}{Point-to-Point Channel}
\newacronym{p2p}{P2P}{Point-to-Point}
\newacronym{ic}{IC}{Interference Channel}
\newacronym{qos}{QoS}{Quality of Service}
\newacronym{wsr}{WSR}{Weighted Sum-Rate}
\newacronym{snr}{SNR}{Signal-to-Noise Ratio}
\newacronym{svd}{SVD}{Singular Value Decomposition}
\newacronym{mmse}{MMSE}{Minimum Mean-Square Error}
\newacronym{wmmse}{WMMSE}{Weighted \gls{mmse}}
\newacronym{mse}{MSE}{Mean-Square Error}
\newacronym{los}{LoS}{Line-of-Sight}
\newacronym{csi}{CSI}{Channel State Information}
\newacronym{cscg}{CSCG}{Circularly Symmetric Complex Gaussian}
\newacronym{sca}{SCA}{Successive Convex Approximation}
\newacronym{kkt}{KKT}{Karush-Kuhn-Tucker}
\newacronym{rf}{RF}{Radio Frequency}

% * page limit hacks
% \captionsetup[subfloat]{farskip=1pt,captionskip=1pt}
% \setlength{\abovecaptionskip}{0pt plus 0pt minus 0pt}

% \setlist{nosep}

% \usepackage{xpatch}
% \setlength{\topsep}{0pt}
% \setlength{\partopsep}{0pt plus 0pt minus 0pt}
% \setlength{\parskip}{0pt}
% \setlength{\parindent}{0pt}
% \makeatletter
% \def\thm@space@setup{%
% 	\thm@preskip=1mm
% 	\thm@postskip=\thm@preskip % or whatever, if you don't want them to be equal
% }
% \xpatchcmd{\proof}{\topsep6\p@\@plus6\p@\relax}{}{}{}
% \makeatother

% \usepackage{titlesec}
% \def\subparagraph{}
% \titlespacing*{\section}{0pt}{0pt}{0pt}
% \titlespacing{\subsection}{0pt}{0pt}{0pt}
% \renewcommand{\thesubsubsection}{\arabic{subsubsection}}
% \titleformat{\subsubsection}[runin]{\itshape}{\thesubsubsection)}{1ex}{}
% \titlespacing*{\subsubsection}{0pt}{0pt}{*1}

% \usepackage{setspace}
% \abovedisplayskip=1mm
% \belowdisplayskip=1mm
% \abovedisplayshortskip=1mm
% \belowdisplayshortskip=1mm
% \setlength{\jot}{0.1mm}
% \setlength{\floatsep}{1mm}
% \setlength{\textfloatsep}{1mm}
% \setlength{\intextsep}{1mm}
% \setlength{\skip\footins}{1mm}
% \addtolength{\parindent}{-1mm}

% * legacy
% % Double superscript
% \newcommand\herm[2][-4]{{#2}^{\mkern#1mu\mathsf{H}}}

% % algpseudocode
% \makeatletter
% \renewcommand{\fnum@algorithm}{\fname@algorithm{} \thealgorithm:}
% \newcommand\setalgorithmcaptionfont[1]{%
% 	\let\my@floatc@ruled\floatc@ruled          % save \floatc@ruled
% 	\def\floatc@ruled{%
% 		\global\let\floatc@ruled\my@floatc@ruled % restore \floatc@ruled
% 		#1\floatc@ruled}}
% \makeatother

\usepackage{response}


\setkomavar{signature}{Yang Zhao, Hongyu Li, Bruno Clerckx, and Massimo Franceschetti}
\setkomavar{date}{\today}
\setkomavar{subject}{Response to Decision on Manuscript T-SP-32387-2024}

\begin{document}
\begin{letter}{%
		Prof. Wei Yi\\
		Associate Editor\\
		IEEE Transactions on Signal Processing
	}
	\opening{Dear Editor and Reviewers,}
	Thank you for giving us the opportunity to submit a revised version of \emph{Channel Shaping Using Beyond Diagonal RIS: Analysis, Optimization, and Enhanced Flexibility}.
	Your comments and suggestions have been invaluable in helping us improve the quality of the manuscript.
	Below we prepare a point-to-point response and highlight the corresponding in-text changes, where labels have been matched to the latest manuscript for your convenience.
	We hope that the revisions and clarifications make the manuscript meet the standards of TSP publications.
	\closing{Yours sincerely,}
\end{letter}


\begin{editor}
	\summary{
		The reviewers raised a number of concerns regarding the paper's contribution, novelty, and mathematical correctness, which led to the recommendation for rejection. The main issues identified include:
		\begin{enumerate}
			\item The contribution of the paper is unclear. Specifically, the proposed framework in \eqref{op:general} is not sufficiently general as it does not consider \gls{bd}-\gls{ris} dependent constraints (e.g., \gls{qos}, sensing quality). A more general case has already been investigated and solved in the authors' previous work \cite{Zhou2023}. Some content in Section \ref{sc:framework} are well established methods in the literature \cite{Abrudan2008}.
			\item The optimization problems \eqref{op:power} and \eqref{op:rate_ris} lacks of novelty. The reviewer pointed that they can be possibly solved by existing methods through appropriate adjustment \cite{Santamaria2023,Li2023b}. Also, it is also pointed out that Corollaries \ref{co:nd_sv_prod_subset} -- \ref{co:nd_sv_prod_tail} are the standard results from matrix textbook.
			\item The correctness and usefulness of the mathematical derivations. For example, the derivation in the proof of Appendix \ref{ap:pareto} is incorrect. Convergence proof in Appendix \ref{ap:power} is questionable. The usefulness of these bounds on singular values of the effective channel $\mathbf{H}$ is unclear.
			\item The literature review can be enhanced to cover a boarder range of papers that related to the problem addressed.
			\item The simulation study can be enhanced. For example, comparisons with the \gls{d}-\gls{ris} and a globally passive \gls{bd}-\gls{ris} were suggested by the reviewer.
		\end{enumerate}
	}

	\reply{
		Good. \cite{Fotock2023}
	}

\end{editor}

\begin{reviewer}
	\summary{
		This manuscript investigates the impact of \gls{bd}-\gls{ris} to the distribution of singular values of \gls{mimo} channel of a point-to-point communication system. Especially, the authors derive bounds of singular values of \gls{bd}-\gls{ris} channel and propose algorithms to design \gls{bd}-\gls{ris} configuration.
	}

	\comment{
		The contribution of Sec. \ref{sc:framework} is unclear for the following two reasons.
		\begin{enumerate}
			\item The manuscript claims to propose a universal optimization framework for \gls{bd}-\gls{ris} as in \eqref{op:general}. However, the framework in \eqref{op:general} is indeed not sufficiently general. It does not consider \gls{bd}-\gls{ris} dependent constraints (e.g., \gls{qos}, sensing quality) besides the orthogonal conditions \eqref{cn:general_unitary}. The reviewer notice that a more general framework, which considers \gls{bd}-\gls{ris} constraints in addition to \eqref{cn:general_unitary}, has already been investigated and solved in the authors' previous work \cite{Zhou2023}.
			\item Besides, both the non-geodesic \gls{rcg} (Sec. \ref{sc:rcg_general}) and geodesic \gls{rcg} (Sec. \ref{sc:rcg_geodesic}) algorithms, e.g., \eqref{eq:geodesic_iden}--\eqref{eq:iterate_mul}, are well established methods in the literature, e.g., \cite{Abrudan2008}. Therefore, the novelty of Sec. \ref{sc:framework} unclear.
		\end{enumerate}
	}

	\reply{
		We appreciate the reviewer's insightful comments. We have revised the manuscript to clarify the contributions of Sec. \ref{sc:framework} and the novelty of the proposed optimization framework. Specifically, we have added a discussion on the limitations of the proposed framework in \eqref{op:general} and the novelty of the proposed geodesic \gls{rcg} method. We have also added a comparison with the existing literature to highlight the novelty of the proposed framework. We believe that the revised manuscript now clearly articulates the contributions of Sec. \ref{sc:framework} and the novelty of the proposed optimization framework.
	}

	\comment{
		In Sec. \ref{sc:pareto}, the manuscript proposes to maximize weighted sum of singular values of the effective channel $\mathbf{H}$. The optimization objective is rather vague. Is the weighted sum of singular values related to any meaningful performance metric (e.g., capacity, power gain, and so on)?
	}

	\comment{
		Derivation in the proof of Lemma \ref{lm:pareto} (Appendix \ref{ap:pareto}) is incorrect. Specifically, $(\partial \mathbf{u}^\mathsf{T}) \cdot \mathbf{u}$ is not equal to $\partial (\mathbf{u}^T \mathbf{u})$. Similar error occurs to $\mathbf{v}$. Please note that the derivative of singular value of a matrix with respect to the matrix itself does not exist. In fact, sub-differential should be used instead, e.g., \cite{Cai2010}.
	}

	\comment{
		The rank equation in the proof of Prop. \ref{pp:dof} (Appendix \ref{ap:dof}) seems confusing. The rank of product of two matrices is: $\rank(\mathbf{AB}) = \rank(\mathbf{B}) - \dim(\ker(\mathbf{A}) \cap \ran(\mathbf{B}))$. How could one obtain the rank equation in Appendix \ref{ap:dof}? Please clarify.
	}

	\comment{
		In Sec. \ref{sc:shaping_analytical}, the manuscript provides a number of bounds on singular values of the effective channel $\mathbf{H}$. However, the usefulness of these bounds is unclear. What can these bounds be used for? Note the results Corollaries \ref{co:nd_sv_prod_subset} -- \ref{co:nd_sv_prod_tail} are quite standard results from matrix textbook. The (upper) bound in Corollary \ref{co:nd_power}, which is achieved by aligning its left and right singular-value vectors to that of the forward and backward channels, respectively, is reminiscent of the results in seminal papers on relay beamforming design, e.g., \cite{Rong2009,Rong2009a}.
	}

	\comment{
		The optimization problems \eqref{op:power} and \eqref{op:rate_ris} lacks of novelty. In fact, they can still be solved by the methods proposed in existing literature, e.g., \cite{Santamaria2023,Li2023b} through appropriate adjustment.
	}

	\comment{
		Convergence proof in Appendix \ref{ap:power} is questionable. First, the manuscript just assumes that solution iterates $\tilde{\mathbf{\Theta}}$ converge. This may not be true (solution iterate could oscillate actually) and is hard to prove. Second, stationary point for constrained problem is more complicated than for the non-constrained case (e.g.  gradient equals zero). Please refer to \cite{Pang2017}.
	}
\end{reviewer}


\begin{reviewer}
	\summary{
	This paper studies the potential of a group-connected \gls{bd}-\gls{ris} to manipulate the \gls{mimo} channel in terms of singular values, power gain, and achievable rate. The analysis and optimization are based on the assumption of asymmetric and lossless \gls{ris} circuit network without mutual coupling. A geodesic design framework is proposed and tested on Pareto frontier shaping and joint beamforming problems. Some analytical bounds on channel singular values and power gain are also provided. Overall, it is a solid work with interesting results. I have some further comments for the authors' reference.
	}

	\comment{
		How practical is it to consider asymmetric reconfigurable \gls{bd}-\gls{ris}? How much gain is expected over symmetric ones and at what cost?
	}

	\comment{
		The geodesic \gls{rcg} method seems promising especially at large group size $L$. Is it possible to extend the design framework to advanced \gls{bd}-\gls{ris} architectures (e.g., multi-sector mode)?
	}

	\comment{
		For fully-connected \gls{bd}-\gls{ris}, the authors mentioned in the result section that power gain-optimal scattering matrix is also rate-optimal. Can you provide a proof on this point?
	}

	\comment{
		The authors refer to \cite{Li2024} for estimating forward and backward channels individually. However, it can be challenging to implement those in real time without RF chains at the \gls{ris}. The impact of imperfect \gls{csi} on the achievable rates may also be studied.
	}

	\comment{
		Some sentences are confusing to me and the presentation can be improved. For example, ``a group-wise geodesic \gls{rcg} method that operates directly on the Stiefel manifold''. Please elaborate more on this.
	}
\end{reviewer}


\begin{reviewer}
	\summary{
		This paper analyzes the channel shaping of a \gls{p2p} \gls{mimo} system, assisted by passive diagonal and/or non-diagonal \glspl{ris} to enhance power and rate gains. Overall, the topic is timely and interesting. Using \gls{ao}, a local optimal solution of the rate maximization for the \gls{bd}-\gls{ris}-assisted \gls{mimo} \gls{p2p} system is obtained. It is shown that \gls{bd}-\gls{ris} improve the achievable rate. To the best of my knowledge, the analysis is solid. However, there are some suggestions for improving the paper further.
	}

	\comment{
		The results for the \gls{p2p} system are insightful. However, a \gls{p2p} system is not typically considered a practical system. It is challenging to extend the results to a system with interfering signals, where the goal is not only to enhance the channel gain of the desired links but also simultaneously to reduce the channel gain of the interfering links. It is more interesting to investigate the rate region of $K$-user \glspl{ic}. It is expected that the authors provide an analysis of the (max-min and sum) rate maximization of multi-user \gls{mimo} systems, assisted by \gls{bd}-\glspl{ris}.
	}

	\comment{
		\gls{bd}-\glspl{ris} provide a greater number of optimization variables but at the expense of increased computational and implementation complexities. While the computational complexities of the proposed solutions are discussed in the paper, a comparison with \gls{d}-\gls{ris} is missing, which should be explicitly addressed in both the introduction and the main body. Additionally, the implementation complexities of \gls{bd}-\glspl{ris}, particularly in comparison to \gls{d}-\gls{ris}, necessitate a more comprehensive analysis. For instance, as highlighted in \cite{Soleymani2024b}, the energy efficiency of a \gls{bd}-\gls{ris} is significantly influenced by the static power consumption of its circuit elements. If the implementation cost of a \gls{bd}-\gls{ris} is substantial, its performance advantages could be considerably reduced.
	}

	\comment{
		The authors considered a passive \gls{ris} architecture. However, another passive configuration, termed \gls{gp}, is discussed in \cite{Fotock2023,Soleymani2024a}. Employing \gls{ao}, the resulting optimization problem to update $\mathbf{\Theta}$ is convex. Comparing the proposed solutions with a globally passive \gls{bd}-\gls{ris} would provide valuable insights into the performance gains obtained by relaxing the (locally) passive assumption. Hence, it is highly recommended to add the \gls{gp} (both \gls{bd} and diagonal) architecture as a benchmark in the numerical results.
	}

% 3) The authors considered a passive RIS architecture. However, another passive configuration, termed globally passive (GP), is discussed in [b], [c]. Employing AO, the resulting optimization problem to update $\Theta$ is convex. Comparing the proposed solutions with a globally passive BD-RIS would provide valuable insights into the performance gains obtained by relaxing the (locally) passive assumption. Hence, it is highly recommended to add the GP (both BD and diagonal) architecture as a benchmark in the numerical results.

\end{reviewer}


\bibliographystyle{IEEEtran}
\bibliography{supplement}
\end{document}
